\documentclass[12pt]{article}
\usepackage{amsfonts, epsfig}

\usepackage{graphicx}
\usepackage{fancyhdr}
\pagestyle{fancy}
\lfoot{\texttt{github.com/conorhoughton/teaching\_misc/i\_and\_f\_workshop}}
\lhead{Python integrate and fire worksheet - Conor}
\rhead{\thepage}
\cfoot{}
\begin{document}

\section*{The integrate and fire model}

The integrate and fire model is a simple model of the voltage dynamics
of a neuron. According to the model, the voltage $V$ satisfies:
\begin{equation}
\tau_m\frac{dV}{dt}=E_L-V+R_mI
\end{equation}
$I$ might end up being synaptic input, but traditionally we write the
equation to match the \textsl{in vivo} experiment where $I$ is an
injected current from an electrode, so we write $I_e$, \lq{}e\rq{} for
electrode. $\tau_m$ is a time constant which physiologically depends
on the membrane resistance and capacitance. 

The equation on its own above leaves out the possibility that there
are other non-linear changes in the currents through the membrane as
$V$ changes. This is a problem since there are other non-linear
changes in the currents through the membrane as $V$ changes. The
equation above leaves these out, in fact, the non-linear effects are
strongest for values of $V$ near where a spike is produced, so one
approach is to use the linear equation unless $V$ reaches a threshold
value and then add a spike \lq{}by hand\rq{}. This has the effect of
changing the voltage to a reset value, this mimics what happens in the
neuron, or in the Hodgkin Huxley model which includes the full
non-linear dynamics which makes the spike. Anyway, in summary
\begin{itemize}
\item $V$ satisfies
\begin{equation}
\tau_m\frac{dV}{dt}=E_L-V+R_mI_e
\end{equation}
\item If $V\ge V_T$ a spike is recorded and the voltage is set to a
  reset value $V_R$.
\end{itemize}
The reset value, the voltage after the spike is often set equal to the
leak potential. This is the \textbf{leaky integrate and fire
  model}. It lacks lots of the details important in the dynamics of
neurons, but is useful and often used for modelling the behaviour of
large neuronal networks or for exploring ideas about neuronal
computation in a relatively straight-forward setting.

\section*{Numerical integration}

In some circumstances it is possible to solve the integrate-and-fire
model analytically, that is, it is possible to write down a
solution. However, it is often necessary to solve it approximately
using a computer. The idea with that is that time is sliced up into
small time steps and the value at each successive time step is
calculated approximately from the previous time step using the
differential equation. This relies on the Taylor expansion:
\begin{equation}
V(t+\delta t)=V(t)+\delta t \frac{dV}{dt}+\frac{1}{2}(\delta t)^2\frac{d^2V}{dt^2}+\ldots
\end{equation}
where the derivatives are evaluated at $t$, not $t+\delta t$. The
\lq{}\ldots\rq{} stands for more terms involving higher and higher
powers of $\delta t$. $\delta t$ in turn is the small time step, which
means that higher and higher powers of $\delta t$ are smaller and
smaller still. The hope is that the derivatives of $V$ don't get bigger and bigger so that we can approximate
\begin{equation}
V(t+\delta t)\approx V(t)+\delta t \frac{dV}{dt}
\end{equation}
This is the Euler approximation, other approximations include more of
the terms, but you can see that it allows you to calculate a value for
$V(t+\delta t)$ from the value at $V(t)$ and the differential equation.

To say this again, but in a way more directly related to how it would
work on a computer, let $t_n=n\delta t$ and $V_n=V(t_n)$ and imagine the differential equation says
\begin{equation}
\frac{dV}{dt}=f(V)
\end{equation}
so in our case $f(V)=[E_l-V+R_mI_e]/\tau_m$. Now the
Euler approximation says that:
\begin{equation}
V_{n+1}=V_n+\delta t f(V_n)
\end{equation}

It is easy to see how that might be calculated as part of a Python
programme. The thing that is missing is the reset; that is easy to add
though, all that is needed is an \texttt{if} statement to check
whether the new value of $V$ exceeds the threshold value $V_T$.

\section*{Programming challenge}

Simulate an integrate and fire model with the following parameters for
one second: $\tau_m = 10 $ms, $E_L = V_r = -70$ mV, $V_t = -40$ mV,
$R_m= 10$ M$\Omega$, $I_e = 3.1 $ nA. Use Euler's method with time-step
$\delta t = 1$ ms. Here $E_L$ is the leak potential, $V_r$ is the
reset voltage, $V_t$ is the threshold, $R_m$ is the membrane
resistance, that is one over the conductance, and $\tau_m$ is the
membrane time constant. Plot the voltage as a function of time. For
simplicity assume that the neuron does not have a refractory period
after producing a spike. You do not need to plot spikes - once
membrane potential exceeds threshold, simply set the membrane
potential to $V_r$.

We haven't spoke about plotting; this is done using
\texttt{matplotlib}; rather than try to describe it there is a sample
programme available in the \texttt{github} folder that plots a sine
function.

\end{document}
