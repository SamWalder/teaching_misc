\documentclass[12pt]{article}
\usepackage{amsfonts, epsfig}

\usepackage{graphicx}
\usepackage{fancyhdr}
\pagestyle{fancy}
\lfoot{\texttt{github.com/conorhoughton/teaching\_misc/i\_and\_f\_workshop}}
\lhead{FitzHugh-Nagumo worksheet - Conor}
\rhead{\thepage}
\cfoot{}
\begin{document}

\section*{The FitzHugh-Nagumo model}
\begin{eqnarray}
\frac{dv}{dt}&=&v-\frac{1}{3}v^3-w+I\cr
\tau\frac{dw}{dt}&=&v+a-bw
\end{eqnarray}
and some good parameters are  $I=0.5$, $a=0.7$, $b=0.8$, and $\tau=12.5$.

\section*{Fourth order Runge-Kutta}

Here are the fourth order Runge Kutta and will include the
possibility that the right hand side of the differential equation also
includes a dependence on $t$ so, writing $t_n=n\delta t$
\begin{equation}
\frac{df}{dt}=G(t,f)
\end{equation}
Now
\begin{eqnarray}
k_1&=&G(t_n,f_n)\cr
k_2&=&G\left(t_n+\frac{1}{2}\delta t,f_n+\frac{1}{2}\delta tk_1\right)\cr 
k_3&=&G\left(t_n+\frac{1}{2}\delta t,f_n+\frac{1}{2}\delta tk_2\right)\cr 
k_4&=&G\left(t_n+\delta t,f_n+\delta tk_3\right) 
\end{eqnarray}
and 
\begin{equation}
f_{n+1}=f_n+\frac{1}{6}(k_1+2k_2+2k_3+k_4)\delta t
\end{equation}
This is written for a single variable but the generalization to more
variables is as you would expect.


\end{document}
